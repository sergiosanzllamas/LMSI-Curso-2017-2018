\documentclass[10pt,a4paper]{article}
\usepackage[utf8]{inputenc}
\usepackage{amsmath}
\usepackage{amsfonts}
\usepackage{amssymb}
\begin{document}

\section{Espacios verticales}
La primera línea separada un espacio pequeño 
\smallskip \\
La linea siguiente \\
La primera línea separada un espacio medio 
\medskip \\
La linea siguiente \\
La tercera linea separada un espacio grande 
\bigskip \\
La linea siguiente \\
La cuarta linea separada un espacio nuevo 
\vspace{8cm} \\
La linea siguiente 

\section{Espacios Horizontales}
Un palabra se separa 2.54 cm 
\hspace{2.54cm} de la siguiente \\
La siguiente frase se 
\hfill{rellena de blancos hasta el final} 

\newpage
\section{Párrafos}
El texto centrado: 
\begin{center}
  La primera línea del párrafo 
  \vspace{2cm} \\ 
  La segunda línea del párrafo 
  \vspace{-0.7cm} \\
  La tercera línea del párrafo \\
\end{center}
El texto a la derecha:
\begin{flushright}
  La primera línea del párrafo \\
  La segunda línea del párrafo \\
  La tercera línea del párrafo \\
\end{flushright}
El texto a la izquierda:
\begin{flushleft}
  La primera línea del párrafo \\
  La segunda línea del párrafo \\
  La tercera línea del párrafo \\
\end{flushleft}
El texto con cita:
\begin{quote}
  - Hola - dijo Pepe. \\
  - Adios - dijo Juan \\
\end{quote}

\section{Tamaños de letra}
Los distintos tamaños son: \\
{\tiny Esto es tamaño tiny} \\
{\scriptsize Esto es tamaño scriptsize} \\
{\footnotesize Esto es tamaño footnotesize} \\
{\small Esto es tamaño small} \\
{\normalsize Esto es tamaño normalsize} \\
{\large Esto es tamaño large} \\
{\Large Esto es tamaño Large} \\
{\LARGE Esto es tamaño LARGE} \\

\end{document}